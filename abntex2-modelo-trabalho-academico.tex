%% abtex2-modelo-trabalho-academico.tex, v-1.9.2 laurocesar
%% Copyright 2012-2014 by abnTeX2 group at http://abntex2.googlecode.com/ 
%%
%% This work may be distributed and/or modified under the
%% conditions of the LaTeX Project Public License, either version 1.3
%% of this license or (at your option) any later version.
%% The latest version of this license is in
%%   http://www.latex-project.org/lppl.txt
%% and version 1.3 or later is part of all distributions of LaTeX
%% version 2005/12/01 or later.
%%
%% This work has the LPPL maintenance status `maintained'.
%% 
%% The Current Maintainer of this work is the abnTeX2 team, led
%% by Lauro César Araujo. Further information are available on 
%% http://abntex2.googlecode.com/
%%
%% This work consists of the files abntex2-modelo-trabalho-academico.tex,
%% abntex2-modelo-include-comandos and abntex2-modelo-references.bib
%%
% ------------------------------------------------------------------------
% ------------------------------------------------------------------------
% abnTeX2: Modelo de Trabalho Academico (tese de doutorado, dissertacao de
% mestrado e trabalhos monograficos em geral) em conformidade com 
% ABNT NBR 14724:2011: Informacao e documentacao - Trabalhos academicos -
% Apresentacao
% ------------------------------------------------------------------------
%Proposta de TCC para Matéria de ITCC. Prof. Carlor Mar.
%Utilizado modelo do abntex2-modelo-trabalho-academico.tex adaptado para a necessidade desta proposta
%pode-se fazer o download deste modelo pelo utilizando o endereço git clone https://github.com/Ricardomguimaraes/tcc-ricardoguimaraes.git.
 
% Ricardo Moraes Guimaraes
% ------------------------------------------------------------------------

\documentclass[
	% -- opções da classe memoir --
	12pt,				% tamanho da fonte
	openright,			% capítulos começam em pág ímpar (insere página vazia caso preciso)
	twoside,			% para impressão em verso e anverso. Oposto a oneside
	a4paper,			% tamanho do papel. 
	% -- opções da classe abntex2 --
	chapter=TITLE,		% títulos de capítulos convertidos em letras maiúsculas
	section=TITLE,		% títulos de seções convertidos em letras maiúsculas
	%subsection=TITLE,	% títulos de subseções convertidos em letras maiúsculas
	%subsubsection=TITLE,% títulos de subsubseções convertidos em letras maiúsculas
	% -- opções do pacote babel --
	english,			% idioma adicional para hifenização
	french,				% idioma adicional para hifenização
	spanish,			% idioma adicional para hifenização
	brazil				% o último idioma é o principal do documento
	]{abntex2}

% ---
% Pacotes básicos 
% ---
\usepackage{lmodern}			% Usa a fonte Latin Modern			
\usepackage[T1]{fontenc}		% Selecao de codigos de fonte.
\usepackage[utf8]{inputenc}		% Codificacao do documento (conversão automática dos acentos)
\usepackage{lastpage}			% Usado pela Ficha catalográfica
\usepackage{indentfirst}		% Indenta o primeiro parágrafo de cada seção.
\usepackage{color}			% Controle das cores
\usepackage{graphicx}			% Inclusão de gráficos
\usepackage{microtype} 			% para melhorias de justificação
% ---
		
% ---
% Pacotes adicionais, usados apenas no âmbito do Modelo Canônico do abnteX2
% ---
\usepackage{lipsum}				% para geração de dummy text
% ---

% ---
% Pacotes de citações
% ---
\usepackage[brazilian,hyperpageref]{backref}	 % Paginas com as citações na bibl
\usepackage[alf]{abntex2cite}	% Citações padrão ABNT

% --- 
% CONFIGURAÇÕES DE PACOTES
% --- 

% ---
% Configurações do pacote backref
% Usado sem a opção hyperpageref de backref
\renewcommand{\backrefpagesname}{Citado na(s) página(s):~}
% Texto padrão antes do número das páginas
\renewcommand{\backref}{}
% Define os textos da citação
\renewcommand*{\backrefalt}[4]{
	\ifcase #1 %
		Nenhuma citação no texto.%
	\or
		Citado na página #2.%
	\else
		Citado #1 vezes nas páginas #2.%
	\fi}%
% ---

% ---
% Informações de dados para CAPA e FOLHA DE ROSTO
% ---
\titulo{Otimização de Rotas na Geração de Pontos Turisticos}
\autor{RICARDO MORAES GUIMARÃES}
\local{Manaus}
\data{2014, 19 de Setembro}
\orientador{Renata da Encarnação Onety}
\coorientador{Anderson Farias Briglia}
\instituicao{%
  Fundação Centro de Análise, Pesquisa e Inovação Tecnológica -- FUCAPI
  \par
  Instituto de Ensino Superior Fucapi
  \par
  Coordenação de Graduação em Ciência da Computação}
\tipotrabalho{TCC (Monografia)}
% O preambulo deve conter o tipo do trabalho, o objetivo, 
% o nome da instituição e a área de concentração 
\preambulo{Modelo canônico de trabalho monográfico acadêmico em conformidade com
as normas ABNT apresentado à comunidade de usuários \LaTeX.}
% ---


% ---
% Configurações de aparência do PDF final

% alterando o aspecto da cor azul
\definecolor{blue}{RGB}{41,5,195}

% informações do PDF
\makeatletter
\hypersetup{
     	%pagebackref=true,
		pdftitle={\@title}, 
		pdfauthor={\@author},
    	pdfsubject={\imprimirpreambulo},
	    pdfcreator={LaTeX with abnTeX2},
		pdfkeywords={abnt}{latex}{abntex}{abntex2}{trabalho acadêmico}, 
		colorlinks=true,       		% false: boxed links; true: colored links
    	linkcolor=blue,          	% color of internal links
    	citecolor=blue,        		% color of links to bibliography
    	filecolor=magenta,      		% color of file links
		urlcolor=blue,
		bookmarksdepth=4
}
\makeatother
% --- 

% --- 
% Espaçamentos entre linhas e parágrafos 
% --- 

% O tamanho do parágrafo é dado por:
\setlength{\parindent}{1.3cm}

% Controle do espaçamento entre um parágrafo e outro:
\setlength{\parskip}{0.2cm}  % tente também \onelineskip

% ---
% compila o indice
% ---
\makeindex
% ---

% ----
% Início do documento
% ----
\begin{document}

% Retira espaço extra obsoleto entre as frases.
\frenchspacing 

% ----------------------------------------------------------
% ELEMENTOS PRÉ-TEXTUAIS
% ----------------------------------------------------------
% \pretextual

% ---
% Capa
% ---
\renewcommand{\imprimircapa}{%
	\begin{capa}%
		\center
		\ABNTEXchapterfont\Large FUNDAÇÃO CENTRO DE ANÁLISE, PESQUISA E INOVAÇÃO TECNOLÓGICA\\ INSTITUTO DE ENSINO SUPERIOR FUCAPI\\ COORDENAÇÃO DE GRADUAÇÃO EM CIÊNCIA DA COMPUTAÇÃO
	
		\vspace*{1cm}
	
		{\ABNTEXchapterfont\large\imprimirautor}
	
		\vfill
		\ABNTEXchapterfont\bfseries\LARGE\imprimirtitulo
		\vfill
	
		\large\imprimirlocal
	
		\large\imprimirdata
	
		\vspace*{1cm}
	\end{capa}
}

\imprimircapa

% ---

% ---
% Folha de rosto
% (o * indica que haverá a ficha bibliográfica)
% ---
\imprimirfolhaderosto*
% ---

% ---
% Inserir a ficha bibliografica
% ---

% Isto é um exemplo de Ficha Catalográfica, ou ``Dados internacionais de
% catalogação-na-publicação''. Você pode utilizar este modelo como referência. 
% Porém, provavelmente a biblioteca da sua universidade lhe fornecerá um PDF
% com a ficha catalográfica definitiva após a defesa do trabalho. Quando estiver
% com o documento, salve-o como PDF no diretório do seu projeto e substitua todo
% o conteúdo de implementação deste arquivo pelo comando abaixo:
%
% \begin{fichacatalografica}
%     \includepdf{fig_ficha_catalografica.pdf}
% \end{fichacatalografica}
%\begin{fichacatalografica}
	%\vspace*{\fill}					% Posição vertical
	%\hrule							% Linha horizontal
	%\begin{center}					% Minipage Centralizado
	%\begin{minipage}[c]{12.5cm}		% Largura
	
	%\imprimirautor
	
	%\hspace{0.5cm} \imprimirtitulo  / \imprimirautor. --
	%\imprimirlocal, \imprimirdata-
	
	%\hspace{0.5cm} \pageref{LastPage} p. : il. (algumas color.) ; 30 cm.\\
	
	%\hspace{0.5cm} \imprimirorientadorRotulo~\imprimirorientador\\
	
	%\hspace{0.5cm}
	%\parbox[t]{\textwidth}{\imprimirtipotrabalho~--~\imprimirinstituicao,
	%\imprimirdata.}\\
	
	%\hspace{0.5cm}
		%1. Palavra-chave1.
		%2. Palavra-chave2.
		%I. Orientador.
		%II. Universidade xxx.
		%III. Faculdade de xxx.
		%IV. Título\\ 			
	
	%\hspace{8.75cm} CDU 02:141:005.7\\
	
	%\end{minipage}
	%\end{center}
	%\hrule
%\end{fichacatalografica}
% ---

% ---
% Inserir errata
% ---
%\begin{errata}
%Elemento opcional da \citeonline[4.2.1.2]{NBR14724:2011}. Exemplo:

%\vspace{\onelineskip}

%FERRIGNO, C. R. A. \textbf{Tratamento de neoplasias ósseas apendiculares com
%reimplantação de enxerto ósseo autólogo autoclavado associado ao plasma
%rico em plaquetas}: estudo crítico na cirurgia de preservação de membro em
%cães. 2011. 128 f. Tese (Livre-Docência) - Faculdade de Medicina Veterinária e
%Zootecnia, Universidade de São Paulo, São Paulo, 2011.

%\begin{table}[htb]
%\center
%\footnotesize
%\begin{tabular}{|p{1.4cm}|p{1cm}|p{3cm}|p{3cm}|}
  %\hline
   %\textbf{Folha} & \textbf{Linha}  & \textbf{Onde se lê}  & \textbf{Leia-se}  \\
    %\hline
    %1 & 10 & auto-conclavo & autoconclavo\\
   %\hline
%\end{tabular}
%\end{table}

%\end{errata}
% ---

% ---
% Inserir folha de aprovação
% ---

% Isto é um exemplo de Folha de aprovação, elemento obrigatório da NBR
% 14724/2011 (seção 4.2.1.3). Você pode utilizar este modelo até a aprovação
% do trabalho. Após isso, substitua todo o conteúdo deste arquivo por uma
% imagem da página assinada pela banca com o comando abaixo:
%
% \includepdf{folhadeaprovacao_final.pdf}
%
%\begin{folhadeaprovacao}

 % \begin{center}
    %{\ABNTEXchapterfont\large\imprimirautor}

    %\vspace*{\fill}\vspace*{\fill}
    %\begin{center}
      %\ABNTEXchapterfont\bfseries\Large\imprimirtitulo
    %\end{center}
    %\vspace*{\fill}
    
    %\hspace{.45\textwidth}
    %\begin{minipage}{.5\textwidth}
        %\imprimirpreambulo
    %\end{minipage}%
    %\vspace*{\fill}
   %\end{center}
        
   %Trabalho aprovado. \imprimirlocal, 24 de novembro de 2012:

   %\assinatura{\textbf{\imprimirorientador} \\ Orientador} 
   %\assinatura{\textbf{Professor} \\ Convidado 1}
   %\assinatura{\textbf{Professor} \\ Convidado 2}
   %\assinatura{\textbf{Professor} \\ Convidado 3}
   %\assinatura{\textbf{Professor} \\ Convidado 4}
      
  % \begin{center}
    %\vspace*{0.5cm}
    %{\large\imprimirlocal}
    %\par
    %{\large\imprimirdata}
    %\vspace*{1cm}
  %\end{center}
  
%\end{folhadeaprovacao}
% ---

% ---
% Dedicatória
% ---
%\begin{dedicatoria}
   %\vspace*{\fill}
   %\centering
   %\noindent
   %\textit{ Este trabalho é dedicado às crianças adultas que,\\
   %quando pequenas, sonharam em se tornar cientistas.} \vspace*{\fill}
%\end{dedicatoria}
% ---

% ---
% Agradecimentos
% ---
%\begin{agradecimentos}
%Os agradecimentos principais são direcionados à Gerald Weber, Miguel Frasson,
%Leslie H. Watter, Bruno Parente Lima, Flávio de Vasconcellos Corrêa, Otavio Real
%Salvador, Renato Machnievscz\footnote{Os nomes dos integrantes do primeiro
%projeto abn\TeX\ foram extraídos de
%\url{http://codigolivre.org.br/projects/abntex/}} e todos aqueles que
%contribuíram para que a produção de trabalhos acadêmicos conforme
%as normas ABNT com \LaTeX\ fosse possível.

%Agradecimentos especiais são direcionados ao Centro de Pesquisa em Arquitetura
%da Informação\footnote{\url{http://www.cpai.unb.br/}} da Universidade de
%Brasília (CPAI), ao grupo de usuários
%\emph{latex-br}\footnote{\url{http://groups.google.com/group/latex-br}} e aos
%novos voluntários do grupo
%\emph{\abnTeX}\footnote{\url{http://groups.google.com/group/abntex2} e
%\url{http://abntex2.googlecode.com/}}~que contribuíram e que ainda
%contribuirão para a evolução do \abnTeX.

%\end{agradecimentos}
% ---

% ---
% Epígrafe
% ---
%\begin{epigrafe}
    %\vspace*{\fill}
	%\begin{flushright}
		%\textit{``Não vos amoldeis às estruturas deste mundo, \\
		%mas transformai-vos pela renovação da mente, \\
		%a fim de distinguir qual é a vontade de Deus: \\
		%o que é bom, o que Lhe é agradável, o que é perfeito.\\
		%(Bíblia Sagrada, Romanos 12, 2)}
	%\end{flushright}
%\end{epigrafe}
% ---

% ---
% RESUMOS
% ---

% resumo em português
%\setlength{\absparsep}{18pt} % ajusta o espaçamento dos parágrafos do resumo
%\begin{resumo}
 %Segundo a \citeonline[3.1-3.2]{NBR6028:2003}, o resumo deve ressaltar o
 %objetivo, o método, os resultados e as conclusões do documento. A ordem e a extensão
 %destes itens dependem do tipo de resumo (informativo ou indicativo) e do
 %tratamento que cada item recebe no documento original. O resumo deve ser
 %precedido da referência do documento, com exceção do resumo inserido no
 %próprio documento. (\ldots) As palavras-chave devem figurar logo abaixo do
 %resumo, antecedidas da expressão Palavras-chave:, separadas entre si por
 %ponto e finalizadas também por ponto.

 %\textbf{Palavras-chaves}: latex. abntex. editoração de texto.
%\end{resumo}

% resumo em inglês
%\begin{resumo}[Abstract]
 %\begin{otherlanguage*}{english}
   %This is the english abstract.

   %\vspace{\onelineskip}
 
   %\noindent 
   %\textbf{Key-words}: latex. abntex. text editoration.
 %\end{otherlanguage*}
%\end{resumo}

% resumo em francês 
%\begin{resumo}[Résumé]
 %\begin{otherlanguage*}{french}
    %Il s'agit d'un résumé en français.
 
   %\textbf{Mots-clés}: latex. abntex. publication de textes.
 %\end{otherlanguage*}
%\end{resumo}

% resumo em espanhol
%\begin{resumo}[Resumen]
 %\begin{otherlanguage*}{spanish}
   %Este es el resumen en español.
  
   %\textbf{Palabras clave}: latex. abntex. publicación de textos.
 %\end{otherlanguage*}
%\end{resumo}
% ---

% ---
% inserir lista de ilustrações
% ---
%\pdfbookmark[0]{\listfigurename}{lof}
%\listoffigures*
%\cleardoublepage
% ---

% ---
% inserir lista de tabelas
% ---
%\pdfbookmark[0]{\listtablename}{lot}
%\listoftables*
%\cleardoublepage
% ---

% ---
% inserir lista de abreviaturas e siglas
% ---
%\begin{siglas}
  %\item[ABNT] Associação Brasileira de Normas Técnicas
  %\item[abnTeX] ABsurdas Normas para TeX
%\end{siglas}
% ---

% ---
% inserir lista de símbolos
% ---
%\begin{simbolos}
  %\item[$ \Gamma $] Letra grega Gama
  %\item[$ \Lambda $] Lambda
  %\item[$ \zeta $] Letra grega minúscula zeta
  %\item[$ \in $] Pertence
%\end{simbolos}
% ---

% ---
% inserir o sumario
% ---
\pdfbookmark[0]{\contentsname}{toc}
\tableofcontents*
\cleardoublepage
% ---



% ----------------------------------------------------------
% ELEMENTOS TEXTUAIS
% ----------------------------------------------------------
\textual

% ----------------------------------------------------------
% Introdução (exemplo de capítulo sem numeração, mas presente no Sumário)
% ----------------------------------------------------------
%Introdução sem numeração, presente no sumario
%\chapter*[Introdução]{Introdução}
%\addcontentsline{toc}{chapter}{Introdução}
\chapter{Introdução}
% ----------------------------------------------------------
As várias opções de cidades com potencial turistico elevado dentro do território brasileiro,
torna a escolha do melhor percuso em uma determinada cidade entre "n" pontos turisticos,
crucial para a sua economia financeira e para a economia do tempo, levando em consideração
o horário de funcionamento do estabelicimento, possibilita ao turista a opção de conhecer
outros lugares dentro do territorio nacional e ainda, a possibilidade de gastos
com outras atividades relacionadas ao turismo.

O problema que o turista tem ao escolher a melhor rota entre "n" pontos para realização do seu percuso
deriva do clássico problema do caixeiro viajante (PVC), onde supõe-se que um caixeiro tenha
de visitar "n" cidades, iniciando e finalizando sua viagem na cidade de origem, com o objetivo de encontrar
a menor rota em toda a viagem. Para sua resolução é necessario conhecer a distância entre as cidades, sendo
quaisquer o sentindo da viagem entre as cidades, o problema é chamado de simétrico.

O problema do caixeiro viajante é uma questão importante da área da otimização matemática com grande relevância nas
áreas de produção, logística e outros. O PVC é modelado através de estruturas demoninadas grafos,
representado por um conjunto de pontos (vértices), podendo ser associados através de linhas,
demoninados (arestas).

Pela perspectiva da otimização o PVC pertence a categoria conhecida como NP-Dificil, possuindo ordem de
complexidade expônencial. Em outras palavras, o número de rotas cresce exponencialmente com o numero de cidades.

Com objetivo de encontrar o melhor caminho entre "n" pontos turisticos, considerando o horário de funcionamento como restrição na otimização do processo, este trabalho utilizará algoritmos Genéticos combinado com o algoritmo 2-opt que realiza um técnica de busca local para encontrar a melhor solução para o problema.


% ----------------------------------------------------------
% PARTE
% ----------------------------------------------------------
%\part{Preparação da pesquisa}
% ----------------------------------------------------------

% ---
% Capitulo com exemplos de comandos inseridos de arquivo externo 
% ---
%\include{abntex2-modelo-include-comandos}
% ---

% ----------------------------------------------------------
% PARTE
% ----------------------------------------------------------
%\part{Referenciais teóricos}
% ----------------------------------------------------------

% ---
% Capitulo de revisão de literatura
% ---
\chapter{Justificativa}
% ---

% ---
\chapter{Objetivos}
\section{Objetivos Geral}
% ---
Desenvolver algoritmo que encontre a melhor rota possível em uma determinada cidade entre "n" pontos
turistico.
% ---
\section{Objetivo Específico}
% ---
\begin{itemize}
\item Definir técnica de otimização que será utilizado para a resolução do problema.
\item Construir uma base de informações com as distâncias entre os pontos turistico juntamente com a restrição horario de funcionamento do ponto turistico.
\item Desenvolver Algoritmo.
\item Avaliar os resultados obtidos.
\end{itemize}

%\lipsum[1]

%\lipsum[2-3]

% ----------------------------------------------------------
% PARTE
% ----------------------------------------------------------
%\part{Resultados}
% ----------------------------------------------------------

% ---
% primeiro capitulo de Resultados
% ---
\chapter{Fundamentação Teórica}
% ---
Serão descritos as técnologias  utilizadas no desenvolvimento da proposta. As tecnologias são: Algoritmos Genéticos,
métodos de busca local.
\section{Algoritmos Genéticos}
Os Algoritmos Genéticos surgiram no momento em que pesquisadores tiveram a necessidade de utilizar computadores para 
realizar a simulação de sistemas biológicos. Porém, depois de vários anos de pesquisa e mais recentemente foi que os algoritmos genéticos foram utlizados em problemas de otimização combinatória passando a ser um tópico de pesquisa. 

Algoritmos Genéticos tem como finalidade, simplificar a formulação de solução dos problemas de otimização, pois a partir de um problema específico implementam uma possivel solução, aplicando operadores de seleção e cruzamento a estruturas que se assemelham a um cromossomo. Um cromossomo, dentro a programação evolutiva é uma população de individuos, onde, matemáticamente um indivíduo representa um solução do problema associado.

Os cromossomos podem  ser representados de várias maneiras, dependendo da classe do problema sua representação pode ser:
\begin{itemize}
\item Binária - Onde os individuos (cromossomos), são codificados por uma sequência binária. Essa representação é utilizada em algoritmos de codificação e decodificação, convertendo a solução para uma sequência binária.
\item Inteiros - Geralmente utilizada na solução de problemas de otimização combinatória, caracteriza-se pela busca de uma solução otima dentre um conjunto de soluções, onde a melhor representação de um individuo é uma vetor de inteiros que representa alguma ordenação de  nós.
\item Real -  Onde os individuos são codificados com uma sequencia de números em ponto flutuante.
\end{itemize}

Os operadores genéticos tem como função, definir regras para um eficaz renovação da população. Há dois tipos de operadores genéticos:
\begin{itemize}
\item Operadores crossover - O primeiro passo na renovação de uma populaçao é gerar novos indivíduos a partir dos ja existentes na população atual. Sua realização é feita pela escolha de pares de indivíduos, chamados de pais, que produzirão dois outros indivíduoss, os filhos, com caracteristicas herdadas dos pais que formarão a nova população.

A idéia do operador crossover é efetuar cruzamento entre dois ou mais cromossomos pais para obter cromossomos filhos, que sao adicionados a população gerando uma nova população.

\item Operadores de mutação - Realizam uma alteração aleatória em cada indivíduo, em todo indivíduo criado na população existirá uma probabilidade próxima a zero de sofrer uma mutação, alterando as caractéristicas herdadas de forma aleatória.
\end{itemize}

\section{O Problema do Caixeiro Viajante - PVC}
O problema do caixeiro viajante tem como objetivo é um problema de roteirização que consiste em encontrar o menor caminho, iniciando em uma cidade de origem, com o dever de visitar todas as cidades pré definidas uma única vez, retornando à cidade de origem.

A aplicabilidade das tecnicas de solução desenvolvidas para o PVC, abrange problemas de lógistica, elêtronica, onde pretende-se encontrar, por exemplo, um caminho minímo para a tarefa de soldar os componentes de uma placa elêtronica. Como mensionado anteriormente, dependendo do tamanho problema o número de rotas pode crescer expônencialmente. 

A resolução do PVC através de algoritmos genéticos se dá pela definição de uma representação para as soluções que seriam os cromossomos, visto na seção anterior, a função de custo que se pretende minimizar e os operadores genéticos que serão utilizados na geração de novas populações. População é o espaço de buca que será exáminado pelo algoritmo genético e cada um dos individuos da população represante uma solução, após ser gerada a população é avaliada pela função aptidão, com o intuíto de medir sua qualidade.

O PVC pertence à classe NP-difícil, pois possui ordem de complexidade exponencial uma vez que sua resolução cresce exponencialmente de acordo com o problema. Na prática, não é possivel encontrar uma solução ótima para o problema, pois os métodos de solução aplicados ao problema, nao garatem uma soluçao ótima.

\section{Heurística de Melhoramento - 2-Opt}
É uma heurística proposta por (CROES, 1958) e (LIN, 1965) e sua principal idéia é eliminar duas aresta de uma solução inicial e inseri-las de forma cruzada, ou seja, se as arestas removidas forem os pares que ligam os pontos  (C1, C2) e (P1, P2), as arestas alteradas ligaram na forma de (C1, P2) e (P1, C2), sendo a nova configuração melhor que a anterior a nova rota é mantida. Se não, é realizada um nova configuração.




% ---
%\section{Vestibulum ante ipsum primis in faucibus orci luctus et ultrices
%posuere cubilia Curae}
% ---

%\lipsum[21-22]

% ---
% segundo capitulo de Resultados
% ---
\chapter{Trabalhos Relacionados}
% ---

% ---
\chapter{Metodologia}
% ---
\chapter{Resultados Esperados}

% ---

\begin{table}[htb]
\ABNTEXfontereduzida
%\label{Cronograma de Atividades para o 2º semestre de 2014}
\begin{tabular}{p{2.6cm} | p{1.5cm} | p{1.5cm} | p{1.6cm} | p{1.5cm}}
  %\hline
   \textbf{Atividade Tempo}  & \textbf{Setembro}  & \textbf{Outubro}  & \textbf{Novembro}  & \textbf{Dezembro}\\
    \hline
    Levantamento\\ Bibliográfico & X &  &  & \\
    \hline
    Desenv. da\\ Proposta &X & X & X &  \\
    \hline
    Desenv. Metod.\\ Investig. & X & X & X & \\
    \hline
    Desenv. do\\ Algoritmo  &  &  & X & X\\
    \hline
    Redação Final\\ Monografia &  &  &  & X\\
    \hline
    Apresentação\\ Monografia &  &  &  & X\\
\end{tabular}
%\legend{Fonte:}
\end{table}

% ---
%\chapter{Resultados Esperados}
% ---

% ---
%\section{Pellentesque sit amet pede ac sem eleifend consectetuer}
% ---

%\lipsum[24]

% ----------------------------------------------------------
% Finaliza a parte no bookmark do PDF
% para que se inicie o bookmark na raiz
% e adiciona espaço de parte no Sumário
% ----------------------------------------------------------
%\phantompart

% ---
% Conclusão (outro exemplo de capítulo sem numeração e presente no sumário)
% ---
%\chapter*[Conclusão]{Conclusão}
%\addcontentsline{toc}{chapter}{Conclusão}
%\chapter{Conlusão}
% ---

%\lipsum[31-33]

% ----------------------------------------------------------
% ELEMENTOS PÓS-TEXTUAIS
% ----------------------------------------------------------
\postextual
% ----------------------------------------------------------

% ----------------------------------------------------------
% Referências bibliográficas
% ----------------------------------------------------------
\bibliography{abntex2-modelo-references}

% ----------------------------------------------------------
% Glossário
% ----------------------------------------------------------
%
% Consulte o manual da classe abntex2 para orientações sobre o glossário.
%
%\glossary

% ----------------------------------------------------------
% Apêndices
% ----------------------------------------------------------

% ---
% Inicia os apêndices
% ---
%\begin{apendicesenv}

% Imprime uma página indicando o início dos apêndices
%\partapendices

% ----------------------------------------------------------
%\chapter{Quisque libero justo}
% ----------------------------------------------------------

%\lipsum[50]

% ----------------------------------------------------------
%\chapter{Nullam elementum urna vel imperdiet sodales elit ipsum pharetra ligula
%ac pretium ante justo a nulla curabitur tristique arcu eu metus}
% ----------------------------------------------------------
%\lipsum[55-57]

%\end{apendicesenv}
% ---


% ----------------------------------------------------------
% Anexos
% ----------------------------------------------------------

% ---
% Inicia os anexos
% ---
%\begin{anexosenv}

% Imprime uma página indicando o início dos anexos
%\partanexos

% ---
%\chapter{Morbi ultrices rutrum lorem.}
% ---
%\lipsum[30]

% ---
%\chapter{Cras non urna sed feugiat cum sociis natoque penatibus et magnis dis
%parturient montes nascetur ridiculus mus}
% ---

%\lipsum[31]

% ---
%\chapter{Fusce facilisis lacinia dui}
% ---

%\lipsum[32]

%\end{anexosenv}

%---------------------------------------------------------------------
% INDICE REMISSIVO
%---------------------------------------------------------------------
%\phantompart
%\printindex
%---------------------------------------------------------------------

\end{document}
